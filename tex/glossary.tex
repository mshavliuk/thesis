% Acronyms and abbreviations
% Syntax: \newacronym{internal_code}{label}{description}

\newacronym{tau}{TAU}{Tampere University}
\newacronym{tuni}{TUNI}{Tampere Universities}
\newacronym{url}{URL}{Uniform Resource Locator}

% Usage: \gls{key}: Prints the acronym. The first time it appears, it prints the long form followed
% by the short form in parentheses. Subsequent uses print the short form only.


%%%%%%%%%%%%%%%%%%%%%%%%%%%%%%%%%%%%%%%%%%%%%%%%%%%%%%%%%%%%%%%%%%%%%%%%%%%%%%%%%%%%%%%%%%%%%%%%%%%%

% Glossary entries
% Syntax: \newglossaryentry{label}{name={term}, description={description}}

\newglossaryentry{iso}{
    name={ISO},
    description={International Organization for Standardization}
}

\newglossaryentry{cc}{
    name={CC licence},
    description={Creative Commons licence}
}

\newglossaryentry{latex}{
    name={\LaTeX},
    description={a document preparation system for scientific writing},
    sort={LaTeX}
}

\newglossaryentry{si}{
    name={SI system},
    description={international system of units (Syst\`eme international d'unit\'es in French)}
}

\newglossaryentry{acceleration}{
name={\ensuremath{\mathbf{a}}},
description={acceleration},
sort={a}
}

\newglossaryentry{force}{
name={\ensuremath{\mathbf{F}}},
description={force},
sort={F}
}

\newglossaryentry{mass}{
name={\ensuremath{m}},
description={mass},
sort={m}
}

\newglossaryentry{reals}
{
name={\ensuremath{\mathbb{R}}},
description={real numbers},
sort={R}
}
